%%%%%%%%%%%%%%%%%%%%%%%%%%%%%%%%%%%%%%%%%%%%%%%%%%%%%%%%%%%%%%%%%%%%%%%%%%%%
\Chapter{The modular isomorphism problem}

Applications of the methods in this package include the study
of the modular isomorphism problems for the groups of small order
from the SmallGroupLibrary - first for groups of order dividing
$2^8$, $3^6$ and $2^9$ \cite{Eic07} \cite{EKo11} and later also $3^7$ and $5^6$
\cite{MM22}.
This section contains the functions used for this purpose as well as an
overview of how the Modular Isomorphism Problem can be studied for any
set of groups using on one hand group-theoretical invariants and on 
the other hand the canonical form of nilpotent algebras. 

%%%%%%%%%%%%%%%%%%%%%%%%%%%%%%%%%%%%%%%%%%%%%%%%%%%%%%%%%%%%%%%%%%%%%%%%%%%%
\Section{Computing bins and checking bins}

A set of groups which share all the group-theoretical invariants implemented
in the package is called *bin*. To determine such bins the main function
available is:

\> BinsByGT( p, n, [L], [false] ) F

If the function is called as $BinsByGT(p, n)$, then it returns
a partition of the list $[1\.\. NumberSmallGroups(p$^$n)]$ into 
sublists so that the groups in the corresponding lists share
all the group-theoretical invariants, i.e. the modular group algebras of two groups 
$SmallGroup(p$^$n, i)$ and $SmallGroup(p$^$n, j)$ over the field
$\F_p$ can not be isomorphic if $i$ and $j$ are in different lists. 

If the function is called as $BinsByGT(p, n, L)$, then $L$ must be a 
list of groups of order $p^n$ and the function will return a partition
of the groups of $L$ which share all the group-theoretical invariants.
Alternatively, $L$ can be a list of group Id's of groups of order $p^n$.

If the function is called as $BinsByGT(p, n, L, false)$ then $L$ must be
a list of groups of order $p^n$ and $false$ disactivates the calculation
of the dimensions of the second cohomology groups. This can be 
switched off as for some type of groups $GAP$ can not apply the needed functions
and since computing the second cohomology groups is arguably the hardest 
of the invariants to test manually.



Several variations of $BinsByGT$ are available. The first two
apply to the case when a list of groups is being studied instead of 
group Id's.

\> MIPSplitGroupsByGroupTheoreticalInvariants ( L ) F

does the same as $BinsByGT(p,n,L)$ but computes the numbers $p$ and
$n$ itself. The input variable must be a list of groups of the same
order. Similarly

\> MIPSplitGroupsByGroupTheoreticalInvariantsNoCohomology(L) F	

computes $BinsByGT(p, n, L, false)$.


Moreover, all the three functions described before have variations
where only those group-theoretical invariants are computed that 
are known to be $\F$-invariants over any field $\F$ of 
characteristic $p$. The input and output of these functions is exactly 
as for the three previous functions.

\> BinsByGTAllFields(p, n , [L], [false]) F

\> MIPSplitGroupsByGroupTheoreticalInvariantsAllFields(L) F

\> MIPSplitGroupsByGroupTheoreticalInvariantsAllFieldsNoCohomology(L) F

The group-theoretical invariants used by the function
$BinsByGT$ and its variations are described below. 
Moreover, $GAP$ prints more or less information on the progress 
inside these functions, if $InfoModIsom$ is set to $1$ or $0$,
respectively. Examples of the use of the functions are included
below.

The main function to apply the algorithm computing the canonical
form of nilpotent algebras in the context of the Modular
Isomorphism Problem is:

\> MIPSplitGroupsByAlgebras( [p, n], bin, [f] ) F

%\> MIPSplitGroupsByAlgebras(bin , [f] ) F

If $MIPSplitGroupsByAlgebras(p, n, bin, [f])$ is called then the algebras
of groups of order $p^n$ with group Id's contained in $bin$ are studied.
The underlying field is of order $p^f$ or, if $f$ is not given, of 
order $p$. 

If the function is called as $MIPSplitGroupsByAlgebras(bin, [f])$ then
$bin$ must be a list of groups of the same prime power order and the function
studies the group algebras of the groups in $bin$ over the field with $p^f$
elements or, if no $f$ is given, of order $p$.

More precisely, in the first case when $bin$ is a list of integers,
for $i \in bin$ let $G_i$ denote $SmallGroup(p$^$n, i)$.
In the second case $G_i$ just runs through the groups contained in $bin$.
Denote by $A_i$ the augmentation ideal of $\F G_i$ where $\F$ is the field
of order $p^f$ or simply $p$, if $f$ is not given.
 The function computes and compares the canonical forms of the algebras 
 $A_i / A_i^j$ for every $i \in bin$ and increasing natural number $j$. 

At each level $j$ it splits the current bins into sub-bins according 
to the different canonical forms of $A_i/A_i^j$. Bins of length 1 are 
then discarded.

The function returns if no further bins are available and provides
information at which level the splitting of the bins took place.

For more evolved calculations one can use the function

\> MIPBinSplit(p, n, max, start, step, L, [f])

Given a list $L$ of small group library Id's or a list of groups of order $p^n$ this functions checks
isomorphism of the associated modular group algebras using canonical forms for the quotients of the
augmentation ideals $A$ of $\F G$. Here $\F$ is either $\F_p$ or $\F_{p^f}$, if $f$ is given.
The parameter $max$ is an integer or $false$ that determines the maximal
quotients $A/A^{max}$ to be checked (if $false$ is given as input, then the quotients are enlarged
until non-isomorphic quotients are found or eventually the full augmentation ideal will be checked).
The parameter $start$ specifies which quotients $A/A^{start}$ are precomputed. The parameter
$step$ determines in which steps the quotients are enlarged if necessary during the isomorphism check.
The output is a record containing three entries: $bins$ contains all the groups, for which the non-
isomorphism of the associated modular group algebras could not be determined; $splits$ contains all
the groups, for which the associated group algebras were determined to be non-isomorphic (and the
first non-isomorphic quotient); $time$ contains the time used for the computation (in milliseconds).

For big algebras all of these function can use a lot of time and memory. To have a
better idea on the progress of the calculations one should set $InfoModIsom$ to 1.

We first show how to study a fixed order:

\beginexample
gap> bins := BinsByGT(2,6);
[ [ 156, 158, 160 ], [ 155, 157 ], [ 173, 176 ], [ 179, 180 ], [ 20, 22 ] ]
gap> List(bins, bin -> MIPSplitGroupsByAlgebras(2, 6, bin));
[ rec( bins := [  ], splits := [ [ 7, [ 156, 158, 160 ] ] ], time := 2195 ), 
  rec( bins := [  ], splits := [ [ 7, [ 155, 157 ] ] ], time := 1505 ), 
  rec( bins := [  ], splits := [ [ 7, [ 173, 176 ] ] ], time := 3294 ), 
  rec( bins := [  ], splits := [ [ 7, [ 179, 180 ] ] ], time := 3233 ), 
  rec( bins := [  ], splits := [ [ 4, [ 20, 22 ] ] ], time := 160 ) ]
\endexample

This shows that the Modular Isomorphism Problem has a positive answer for groups
of order $64$ for the field $\F_2$. The result means e.g. that the smallest quotients
(of Loewy layers)
such that the augmentation ideals $A_1$ and $A_2$ of the groups algebras over $\F_2$ of the
groups $SmallGroup(64, 156)$ and $SmallGroup(64, 158)$ are not isomorphic are $A_1/A_1^8$
and $A_2/A_2^8$. These are $A_1/A_1^5$ and $A_2/A_2^5$ for the groups
$SmallGroup(64, 20)$ and $SmallGroup(64, 22)$. 

The following shows that the problem also has a positive answer for the group
algebras of groups of order $64$ over the field $\F_4$. Note that for the groups
$SmallGroup(64, 20)$ and $SmallGroup(64, 22)$ one has to consider deeper quotients
in this case.

\beginexample
gap> bins := BinsByGTAllFields(2,6);
[ [ 156, 158, 160 ], [ 155, 157 ], [ 173, 176 ], [ 179, 180 ], [ 104, 105 ], 
  [ 13, 14 ], [ 20, 22 ], [ 18, 19 ] ]
gap> List(bins, bin -> MIPSplitGroupsByAlgebras(2, 6, bin, 2));
[ rec( bins := [  ], splits := [ [ 7, [ 156, 158, 160 ] ] ], time := 34833 ), 
  rec( bins := [  ], splits := [ [ 7, [ 155, 157 ] ] ], time := 22479 ), 
  rec( bins := [  ], splits := [ [ 7, [ 173, 176 ] ] ], time := 9806 ), 
  rec( bins := [  ], splits := [ [ 7, [ 179, 180 ] ] ], time := 7819 ), 
  rec( bins := [  ], splits := [ [ 4, [ 104, 105 ] ] ], time := 2226 ), 
  rec( bins := [  ], splits := [ [ 6, [ 13, 14 ] ] ], time := 707 ), 
  rec( bins := [  ], splits := [ [ 6, [ 20, 22 ] ] ], time := 3917 ), 
  rec( bins := [  ], splits := [ [ 6, [ 18, 19 ] ] ], time := 2891 ) ]

\endexample

The other functions allow to study the problem for groups not coming from the
library. The following groups are studied in \cite{GLM24}.

\beginexample
R := SmallGroup(64, 19);
Q := SmallGroup(64, 18);

DR := DirectProduct(R,Q);
GDR := GeneratorsOfGroup(DR);
z1 := GDR[3];
z2 := GDR[9];
N := Group(z1*z2^(-1));
G := DR/N;

DR := DirectProduct(Q,Q);
GDR := GeneratorsOfGroup(DR);
z1 := GDR[3];
z2 := GDR[9];
N := Group(z1*z2^(-1));
H := DR/N;

gap> MIPSplitGroupsByGroupTheoreticalInvariantsAllFields([G,H]);
[ [ Group([ f1, f2, f7, f3, f4, f10, f5, f6, f7, f8, f9, f10 ]), 
      Group([ f1, f2, f7, f3, f4, f10, f5, f6, f7, f8, f9, f10 ]) ] ]

# the groups can not be split over all fields by group-theoretical invariants

gap> MIPSplitGroupsByAlgebras([G,H]);
rec( bins := [  ], 
  splits := 
    [ 
      [ 4, 
          [ Group([ f1, f2, f7, f3, f4, f10, f5, f6, f7, f8, f9, f10 ]), 
              Group([ f1, f2, f7, f3, f4, f10, f5, f6, f7, f8, f9, f10 ]) ] ] 
     ], time := 44473 )
     
# over the field of 2 elements it is enough to consider
# the 5-th power of the augmentation ideal

\endexample

The program does not finish in a very reasonable time, if we run it over the field $\F_4$, but we
can still check that it is not enough to factor out only the 5th power of the augmentation ideal
in this case. One option is to use info level to do this and the other to use $MIPBinsSplit$:

\beginexample
gap> SetInfoLevel(InfoModIsom, 1);
gap> MIPSplitGroupsByAlgebras([G,H], 2);
#I  Refine bin
#I    Weights yields bins [ [ 1, 2 ] ]
#I    Layer 1 yields bins [ [ 1, 2 ] ]
#I  layer 2 of dim 15 aut group has order 2961100800 * 2^0
#I     cover is determined 
#I     dim(M) = 16 and dim(U) = 5
#I     extended autos 
#I     computed stabilizer
#I     got quotient 
#I     induced autos 
#I  layer 2 of dim 15 aut group has order 2961100800 * 2^0
#I     cover is determined 
#I     dim(M) = 16 and dim(U) = 5
#I     extended autos 
#I     computed stabilizer
#I     got quotient 
#I     induced autos 
#I    Layer 2 yields bins [ [ 1, 2 ] ]
#I  layer 3 of dim 39 aut group has order 2937600 * 2^88
#I     cover is determined 
#I     dim(M) = 29 and dim(U) = 5
#I     extended autos 
#I     computed stabilizer
#I     got quotient 
#I     induced autos 
#I  layer 3 of dim 39 aut group has order 2937600 * 2^88
#I     cover is determined 
#I     dim(M) = 29 and dim(U) = 5
#I     extended autos 
#I     computed stabilizer
#I     got quotient 
#I     induced autos 
#I    Layer 3 yields bins [ [ 1, 2 ] ]
#I  layer 4 of dim 81 aut group has order 2937600 * 2^240
#I     cover is determined 
#I     dim(M) = 51 and dim(U) = 9
#I     extended autos 
#I     computed stabilizer
#I     got quotient 
#I     induced autos 
#I  layer 4 of dim 81 aut group has order 2937600 * 2^240
#I     cover is determined 
#I     dim(M) = 51 and dim(U) = 9
#I     extended autos 
#I     computed stabilizer
#I     got quotient 
#I     induced autos 
#I    Layer 4 yields bins [ [ 1, 2 ] ]
#I  layer 5 of dim 145 aut group has order 7200 * 2^496
#I     cover is determined 
#I     dim(M) = 73 and dim(U) = 9
#I     extended autos 
#I     computed stabilizer
#I     got quotient 
#I     induced autos 
#I  layer 5 of dim 145 aut group has order 7200 * 2^496
#I     cover is determined 
#I     dim(M) = 73 and dim(U) = 9
#I     extended autos 
#I     computed stabilizer
#I     got quotient 
#I     induced autos 
#I    Layer 5 yields bins [ [ 1, 2 ] ]
#I  layer 6 of dim 231 aut group has order 7200 * 2^800
#I     cover is determined 
#I     dim(M) = 95 and dim(U) = 9
#I     extended autos 
#I     computed stabilizer
#I     got quotient 
^CError, user interrupt in
  AddRowVector( u, GetEntryTable( T, i, j ), v[i] * w[j] 
 ); at /home/leo/gap-4.10.1/pkg/modisom-2.5.3/gap/tables/tables.gi:87 called from 
MultByTable( Q, new[Q.wds[i][1]], new[Q.wds[i][2]] 
 ) at /home/leo/gap-4.10.1/pkg/modisom-2.5.3/gap/autiso/induc.gi:135 called from
InduceAutoToQuot( Q, G.agAutos[i] 
 ) at /home/leo/gap-4.10.1/pkg/modisom-2.5.3/gap/autiso/induc.gi:151 called from
InduceAutosToQuot( G, Q 
 ); at /home/leo/gap-4.10.1/pkg/modisom-2.5.3/gap/autiso/autiso.gi:57 called from
ExtendCanoForm( tabs[i], j 
 ); at /home/leo/gap-4.10.1/pkg/modisom-2.5.3/gap/grpalg/chkbins.gi:117 called from
MIPBinSplit( p, n, false, start, step, list, f 
 ) at /home/leo/gap-4.10.1/pkg/modisom-2.5.3/gap/grpalg/chkbins.gi:177 called from
...  at *stdin*:39
you can 'return;'
brk> quit;   # this was not progressing for several hours

gap> Size(G) = Size(H);
true
gap> Size(G) = 2^10;
true

gap> MIPBinSplit(2, 10, 4, 4, 1, [G,H], 2);
#I  Refine bin
#I    Weights yields bins [ [ 1, 2 ] ]
#I    Layer 1 yields bins [ [ 1, 2 ] ]
#I  layer 2 of dim 15 aut group has order 2961100800 * 2^0
#I     cover is determined 
#I     dim(M) = 16 and dim(U) = 5
#I     extended autos 
#I     computed stabilizer
#I     got quotient 
#I     induced autos 
#I  layer 2 of dim 15 aut group has order 2961100800 * 2^0
#I     cover is determined 
#I     dim(M) = 16 and dim(U) = 5
#I     extended autos 
#I     computed stabilizer
#I     got quotient 
#I     induced autos 
#I    Layer 2 yields bins [ [ 1, 2 ] ]
#I  layer 3 of dim 39 aut group has order 2937600 * 2^88
#I     cover is determined 
#I     dim(M) = 29 and dim(U) = 5
#I     extended autos 
#I     computed stabilizer
#I     got quotient 
#I     induced autos 
#I  layer 3 of dim 39 aut group has order 2937600 * 2^88
#I     cover is determined 
#I     dim(M) = 29 and dim(U) = 5
#I     extended autos 
#I     computed stabilizer
#I     got quotient 
#I     induced autos 
#I    Layer 3 yields bins [ [ 1, 2 ] ]
#I  layer 4 of dim 81 aut group has order 2937600 * 2^240
#I     cover is determined 
#I     dim(M) = 51 and dim(U) = 9
#I     extended autos 
#I     computed stabilizer
#I     got quotient 
#I     induced autos 
#I  layer 4 of dim 81 aut group has order 2937600 * 2^240
#I     cover is determined 
#I     dim(M) = 51 and dim(U) = 9
#I     extended autos 
#I     computed stabilizer
#I     got quotient 
#I     induced autos 
#I    Layer 4 yields bins [ [ 1, 2 ] ]
rec( 
  bins := 
    [ 
      [ Group([ f1, f2, f7, f3, f4, f10, f5, f6, f7, f8, f9, f10 ]), 
          Group([ f1, f2, f7, f3, f4, f10, f5, f6, f7, f8, f9, f10 ]) ] ], 
  splits := [  ], time := 9469981 )

\endexample

%%%%%%%%%%%%%%%%%%%%%%%%%%%%%%%%%%%%%%%%%%%%%%%%%%%%%%%%%%%%%%%%%%%%%%%%%%%%%%%%%%%
\Section{Kernel size}

An idea to study the Modular Isomorphism Problem is to define maps on certain
quotients of the augmentation ideal $A$ and count the number of elements which map
to $0$ under this map. The map most typically used for this is a $p$-power map
$A^n/A^{n+m} \rightarrow A^{n \cdot p^\ell}/A^{n \cdot p^\ell + m}$. This can 
be done in the package using the function

\> KernelSizePowerMap(T, n, m, l, [f]) F

where $T$ is a table as returned by $ModIsomTable$ and $n, m, l$ are as 
just described and the calculation is performed over the field $\F_{p^f}$.
If $f$ is not given, then it is set to $1$.
We can check for instance the first calculation in \cite{HS06}(Section 4.1).

\beginexample
gap> G := SmallGroup(64, 20);
<pc group of size 64 with 6 generators>
gap> H := SmallGroup(64, 22);
<pc group of size 64 with 6 generators>
gap> TG := ModIsomTable(G, 5);;
gap> TH := ModIsomTable(H, 5);;
gap> KernelSizePowerMap(TG, 1, 1, 2);
3
gap> KernelSizePowerMap(TH, 1, 1, 2);
1
\endexample

This shows that the group algebras over $\F_2$ are not isomorphic. This 
argument does however not work over $\F_4$:

\beginexample
gap> TG := ModIsomTable(G, 5, 2);;
gap> TH := ModIsomTable(H, 5, 2);;
gap> KernelSizePowerMap(TG, 1, 1, 2);
7
gap> KernelSizePowerMap(TH, 1, 1, 2);
7
\endexample

%%%%%%%%%%%%%%%%%%%%%%%%%%%%%%%%%%%%%%%%%%%%%%%%%%%%%%%%%%%%%%%%%%%%%%%%%%%%%
\Section{The group theoretical invariants}

We document here which group-theoretical invariants are used in $BinsByGT$ and 
similar functions.

\> GroupInfo(G) F

This is an auxiliary function used in other group-theoretical invariants. 
If $IdGroup$ is available in $GAP$ for the order of $G$ it returns $IdGroup(G)$.
Otherwise it returns $[Size(G), AbelianInvariants(G)]$. 

We now describe the invariants in the order they appear in $BinsByGT$. 
First the isomorphism types of $G/G'$ and $Z(G)$, the abelianization and the center
of $G$ are used. These are very classical invariants \cite{San85}(Theorems 6.12, 6.7).
We next list the other functions which are applied, which are all small functions written 
for the package:

\> CenterDerivedInfo(G) F

calculates the isomorphism types of $Z(G) \cap G'$ and $Z(G)/ Z(G) \cap G'$ 
\cite{San85}(Theorem 6.11).

\> SandlingInfo(G) F

calculates several invariants coming from the small groups algebra which was first
used to study the Modular Isomorphism Problem in \cite{San89}. Namely, for $\gamma_i(G)$
the $i$-th term of the lower central series of $G$ it returns the $GroupInfo$ for
$G/\gamma_2(G)^p\gamma_3(G)$ \cite{San89}, the $GroupInfo$ of $G/\gamma_2(G)^p\gamma_4(G)$,
if $G$ is $2$-generated (mentioned in \cite{Bag99}, proved in \cite{MM22}) and if the
derived subgroup of $G$ is elementary abelian and the Jennings series of $G$ has length
at most $2p$, it returns $GroupInfo$ for the Frattini subgroup of $G$ \cite{HS06}(p.16).
This function is not applied in $BinsByGTAllFields$ and its variations.

\> JenningsInfo(G) F

Denoting by $D_i(G)$ the $i$-th member of the Jennings series, this function returns   
$GroupInfo$ for the different quotients $D_i(G)/D_{i+1}(G)$, $D_i(G)/D_{i+2}(G)$, $D_i(G)D_{2i+1}(G)$
for meaningful values of $i$ (results \cite{Jen41} \cite{PS72} \cite{RS83}) and if $p$ is odd also for
$G/D_4(G)$ \cite{Her07}. For $BinsByGTAllFields$ and its variations only the quotients
$D_i(G)/D_{i+1}(G)$ are computed.

\> JenningsDerivedInfo(G) F

computes $D_i(G')/D_{i+1}(G')$ for all $i$ \cite{San85}(Lemma 6.26).

\> BaginskiInfo(G) F

For $N = C_G(G'/\Phi(G'))$, where $\Phi(G)$ denotes the Frattini subgroup, if $G/N$ is
cyclic, it returns the $GroupInfo$ for $N/\Phi(G')$ and $G/\Phi(N)$ \cite{Bag99}.
This function is not applied in $BinsByGTAllFields$ and its variations.

\> BaginskiCarantiInfo(G) F

returns the nilpotency class of $G/\Phi(G')$ \cite{BC88}(Proposition 2.1). 
This function is not applied in $BinsByGTAllFields$ and its variations.

\> NilpotencyClassInfo(G) F

returns the nilpotency class of $G$, when the exponent of $G$ is $p$ or the class equals 2
or the derived subgroup is cyclic \cite{BK07}.

\> Theorem41MS22(G) F

If $p$ is odd, $G$ is $2$-generated, the nilpotency class of $G$ is $3$ and $\gamma_3(G)$ 
has exponent $p$, it returns the isomorphism types of $\gamma_2(G)$ and $\gamma_3(G)$
\cite{MS22}(Theorem 4.1). This function is not applied in $BinsByGTAllFields$ and its variations.

\> CyclicDerivedInfo(G) F

If $p$ is odd and $G'$ is cyclic this returns several invariants contained in \cite{GLdRS22} \cite{GLdR23}.
Namely, the quotients $D_i(C_G(G'))/D_{i+1}(C_G(G'))$ for all $i$, the exponent of $C_G(G')$, the isomorphism
type of $C_G(G')/G'$ and the $GroupInfo$ for $G/R_1(\gamma_3(G))$ and $G/R_3(C_G(G'))$.
Here $R_i(G)$ denotes the subgroup of $G$ generated by the $p^i$-th powers in $G$.
If $G$ is additionally $2$-generated, it also computes the $GroupInfo$ for $C_G(G')$ and the
type invariants of $G$ (cf. \cite{GLdRS22} for the definition). 
For $BinsByGTAllFields$ and its variations, the $GroupInfo$ for  $G/R_1(\gamma_3(G))$
 and $G/R_3(C_G(G'))$ is not computed.
 
\> MaximalAbelianDirectFactor(G) F

computes the maximal abelian direct factor of $G$ \cite{GL24}. In $BinsByGTAllFields$ and
its variations it computes the maximal elementary abelian direct factor instead \cite{MSS23}.

\> NormalSubgroupsInfo(G) F

computes some of the sections of $G$ which are canonical following \cite{GL24}(Lemma 3.6). The starting canonical group is the derived subgroup.

\> IsCoveredByTheory(G) F

determines, whether $G$ belongs to certain classes for which the Modular Isomorphism
Problem has been solved positively. Namely if $G$ is a $2$-group of maximal class \cite{Bag92},
$p$ is odd, $G \leq p^{p+1}$, $G$ has a maximal abelian subgroup and $G$ is of maximal class \cite{BC88},
$[G:Z(G)] = p^2$ \cite{Dre89}, $G$ is a $2$-generated $2$-group of nilpotency class $2$ \cite{BdR21},
$p$ is odd, $G'$ is elementary abelian, the nilpotency class of $G$ is $3$ and either 
$C_G(G')$ is a maximal subgroup of $G$ and abelian \cite{MS22}(Theorem 3.3) or a $G$ is
$3$-generated and a certain condition holds on the commutator map modulo the second center of $G$
\cite{MS22}(Theorem 3.5), $G$ is a $2$-group with cyclic center and dihedral central
quotient \cite{MSS23}, $G$ is a $2$-group of nilpotency class $2$ with cyclic center \cite{GLM24},
$p$ is odd, $G'$ is cyclic and $R_2(G/G')$ is cyclic \cite{GLdR23}(Proposition 3.7). 
For $BinsByGTAllFields$ and its variations it only checks, if $G$ is a $2$-group of
maximal class, $[G:Z(G)] = p^2$ or $G$ is a $2$-group with cyclic center and dihedral central
quotient.

\> DimensionTwoCohomology(G) F

computes the dimensions of the second cohomology group $H^2(FG, F)$ and of the second
Hochschild cohomology group $HH^2(FG) = H^2(FG, FG)$.

\> ConjugacyClassInfo(G) F

Computes the number of conjugacy classes which are $p^i$-th powers, for all $i$ (Kuelshammer) and
the number of conjugacy classes of $p^i$-th powers which come from conjugacy classes of the same order
 (Parmenter-Polcino Milies) \cite{HS06}(Section 2.2) and the dimension of the first Hochschild 
 cohomology group which equals the number $\sum_{g^G} \log_p(C_G(g)/\Phi(C_G(g))$ where the sum runs over
 all the conjugacy classes of $G$ \cite{HS06}(Section 2.6). 
 
 The return contains first the Roggenkamp parameter, then the Kuelshammer invariants (starting from $i = 0)$ and then the invariant of Parmenter-Polcino Milies.
 
\> SubgroupsInfo(G) F

Computes the number of conjugacy classes of maximal elementary abelian subgroups of rank 1,2,...
The return is a list of integers which contains this number for all ranks until the maximal possible.
This is based on results of Quillen \cite{HS06}(Section 2.5).


